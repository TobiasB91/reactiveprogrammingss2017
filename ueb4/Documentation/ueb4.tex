\documentclass{rp}

\begin{document}

%% Daten der Übungsgruppe (am besten in separater Datei defs.tex ablegen,
%% und mit \input{defs} immer wieder einbinden):

% Name des Tutors:
\tutor{Martin Ring}
% Nummer der Übungsgruppe:
\uebungsgruppe{1} 
% Teilnehmer (bis zu drei, mit '\\' getrennt):
\teilnehmer{Tobias Brandt \\ Stefan Heitmann \\ }

\maketitle{4}          % Übungsblatt-Nr.
{19.06.2017} % Abgabe-Datum

\exercise{Curried in Space}%
\\
\textbf{Server}\\
Zunächst wurden alle Nachrichtentypen erstellt und ihre entsprechende Kodierung als JSON-Objekt implementiert.
\IncludeFun{22}{120}{../haskell-server/Messages.hs}

Danach wurde auf dem Server ein Connection-Aktor erstellt. Dieser repräsentiert jeweils eine Verbindung mit einem Client. 
\IncludeFun{11}{33}{../haskell-server/Connection.hs}

Anschließend wurde ein Universums-Aktor erstellt. Dieser verwaltet alle anderen Objekte in der Simulation. 
\IncludeFun{15}{115}{../haskell-server/Universe.hs}

Alle weiteren Objekte werden ebenfalls durch Aktoren dargestellt, die jeweils bei einer Update-Nachricht ihre neue Position im Raum berechnen und an das Universum senden. 
\IncludeFun{118}{293}{../haskell-server/Universe.hs}
\\
\textbf{Client}\\
Im Client wurden zuerst Objekte um einen Identifikation erweitert, sodass die Objekte einfacher wiedergefunden werden können. 
\IncludeFun{7}{184}{../scala-client/src/main/scala/SpaceObjects.scala}

Zum Wiederauffinden wurde eine Look-up-Methode implementiert.
\IncludeFun{66}{68}{../scala-client/src/main/scala/SpaceView.scala}

Ebenso wurde auch im Client die Nachrichten implementiert, die bereits auf dem Server implementiert wurden. 
\IncludeFun{9}{101}{../scala-shared/shared/src/main/scala/Messages.scala}

Die eintreffenden Objekte erneuern die Werte für ihre entsprechende Repräsentation auf dem Client. Sollte keine entsprechende Repräsentation vorhanden sein, wird ein neue Repräsentation erzeugt. 
\IncludeFun{41}{165}{../scala-client/src/main/scala/Main.scala}


\textbf{Tests}\\
Getestet wurden beide Aufgaben, indem die Simulation funktional ausprobiert wurde. Die Tests verliefen erfolgreich.
\end{document}

