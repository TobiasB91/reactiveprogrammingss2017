\documentclass{rp}

\begin{document}

%% Daten der Übungsgruppe (am besten in separater Datei defs.tex ablegen,
%% und mit \input{defs} immer wieder einbinden):

% Name des Tutors:
\tutor{Martin Ring}
% Nummer der Übungsgruppe:
\uebungsgruppe{1} 
% Teilnehmer (bis zu drei, mit '\\' getrennt):
\teilnehmer{Tobias Brandt \\ Stefan Heitmann \\ }

\maketitle{2}          % Übungsblatt-Nr.
{11.05.2017} % Abgabe-Datum

\exercise{Promise: My Future in Haskell}
Ein Promise ist in Haskell nur ein Halter für ein Future. Ein Future hält eine MVar und eine mögliche Aktion.
\IncludeFun{5}{6}{../exercise1/Future.hs}

Der Rest der Aufgabe wurde nach Spezifikation implementiert. 
\IncludeFun{8}{28}{../exercise1/Future.hs}

Zum Testen wurde das mitgelieferte Roboter-Programm ausgeführt. Es verhielt sich ähnlich zu dem in der Vorlesung gezeigten Scala-Implementierung.

\exercise{MVar in Scala?!}
Die Schnittstellen könnten in Scala ungefähr so aussehen. 
\begin{lstlisting}[language=scala]
class MVar[A] {
    var value = null
    def takeMVar : A = { ... }
    def putMVar(a : A) : Unit = { ... }
}
\end{lstlisting}

\exercise{6 nimmt!}
Wir habens uns dazu entschlossen, die gesamte Funktionalität auf dem Server zu realisieren. Sodass Programme wie \textit{Telnet} als Client verwendet werden können.\\\\

In der Funktion \texttt{startServer} wird ein Socket geöffnet. Anschließend wird bereits ein Thread für die Handhabung des Spielens gestartet (\texttt{newGame}). Danach wird in diesem Thread nur noch auf ankommende Verbindungen gewartet. 
\IncludeFun{29}{43}{../exercise2/takeFive/src/Server.hs}

Sollte eine neue Verbindung eingehen, wird ein neuer Thread gestartet, der für die Identifizierung einer Person zuständig ist. Dieser erfragt einen Namen von der eigehenden Verbindung. Sollte der eingegangene Name valide sein, wird diese Verbindung als neuer Spieler in die Spielerlisten-MVar eingetragen. Zusätzlich wird der Spieler als ''noch nicht bereit'' markiert. 
\IncludeFun{45}{63}{../exercise2/takeFive/src/Server.hs}

Zusätzlich wird für jeden akzeptierten Client jeweils ein Thread zum Einlesen von Eingaben und zum Ausgeben vom Nachrichten vom Server gestartet. Die Kommunikation wird dabei über Channels realisiert. Damit benutzen wir MVars indirekt, da Channels \textit{''lediglich''} einen Stream von MVars darstellen. Die Channels werden dabei immer von dem Wurzel-Eingabe- bzw. Wurzel-Ausgabe-Channel dupliziert, welche ebenfalls in dem Spiele-Thread bekannt sind. 
\IncludeFun{65}{77}{../exercise2/takeFive/src/Server.hs}

Der Spiele-Thread besteht aus zwei Phasen.
\IncludeFun{79}{82}{../exercise2/takeFive/src/Server.hs}

Die erste Phase überprüft, ob alle Spieler in der Spielerlisten-MVar als ''bereit'' markiert sind. Sollte dies der Fall sein, endet die erste Phase und die zweite Phase beginnt. Zusätzlich stellt die erste Phase Kommandos für den Client bereit, um Informationen über die anderen Spieler zu ergattern. 
\IncludeFun{84}{103}{../exercise2/takeFive/src/Server.hs}

In der zweiten Phase findet das eigentliche Spiel statt. Dazu haben wir zunächst das Spielfeld und die Spieler als algebraische Datentypen dargestellt. 
\IncludeFun{23}{27}{../exercise2/takeFive/src/Server.hs}

Außerdem haben wir eine Reihe an Funktionen geschrieben, die den entsprechenden Spielzustand an die Clients ausgibt. 
\IncludeFun{147}{170}{../exercise2/takeFive/src/Server.hs}

Die \texttt{game}-Funktion initialisiert das Spiel, startet es und spielt solange Spielrunden bis ein Spieler mindestens 66 Minuspunkte gesammelt hat. 
\IncludeFun{105}{145}{../exercise2/takeFive/src/Server.hs}

Eine einzelne Spielrunde beginnt durch das Auswählen der Karten durch die Spieler. \texttt{selectCards} liest die Eingaben der Spieler aus dem Eingabe-Channel aus und überprüft, ob die Eingabe valide war. Sollten alle Spieler valide Eingaben getätigt haben, wird das Einsortieren gestartet. 
\IncludeFun{172}{198}{../exercise2/takeFive/src/Server.hs}

Das Einsortieren wird dann in der (etwas langen) Funktion \texttt{sortIn} realisiert. 
\IncludeFun{200}{251}{../exercise2/takeFive/src/Server.hs}

Sollte eine Karte so niedrig sein, dass sie nirgends auf dem Tisch einsortiert werden kann, wird der User in durch die Funktion \texttt{selectRow} befragt, welche Reihe der User auf seinen Hornochsenstapel nehmen möchte. 
\IncludeFun{253}{277}{../exercise2/takeFive/src/Server.hs}

Das Programm wurde funktional getestet, indem mehrere Partien gespielt wurden. 
\end{document}

