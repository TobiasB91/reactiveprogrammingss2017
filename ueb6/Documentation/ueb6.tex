\documentclass{rp}

\begin{document}

%% Daten der Übungsgruppe (am besten in separater Datei defs.tex ablegen,
%% und mit \input{defs} immer wieder einbinden):

% Name des Tutors:
\tutor{Martin Ring}
% Nummer der Übungsgruppe:
\uebungsgruppe{1} 
% Teilnehmer (bis zu drei, mit '\\' getrennt):
\teilnehmer{Tobias Brandt \\ Stefan Heitmann \\ }

\maketitle{6}          % Übungsblatt-Nr.
{06.07.2017} % Abgabe-Datum

\exercise{Hoogle Docs}%
\\
\textbf{Server}\\
messages und serialisierung
\IncludeFun{9}{48}{../app/Messages.hs}
\\\\
server cursor
\IncludeFun{21}{35}{../app/Main.hs}
\\\\
server cursor2
\IncludeFun{37}{67}{../app/Main.hs}
\\\\
server cursor3
\IncludeFun{77}{91}{../app/Main.hs}
\\\\
ot
\IncludeFun{116}{139}{../src/OT.hs}
\\\\
ot2
\IncludeFun{16}{18}{../src/OT.hs}
\\\\
ot3
\IncludeFun{65}{75}{../src/OT.hs}
\\\\
ot4
\IncludeFun{22}{63}{../src/OT.hs}
\\\\
\textbf{Client}\\
move cursor bei actions
\IncludeFun{11}{41}{../client/src/main/scala/Editor.scala}
\\\\
empfangen von edits und mehr :) 
\IncludeFun{58}{106}{../client/src/main/scala/Main.scala}
\\\\
Transformcursor und apply op
\IncludeFun{117}{151}{../client/src/main/scala/OT.scala}
\\\\
rendern der cursor angepasst
\IncludeFun{59}{77}{../client/src/main/scala/Editor.scala}
\\\\
compose und transform
\IncludeFun{35}{115}{../client/src/main/scala/OT.scala}
\\\\
reduce der inserts
\IncludeFun{174}{180}{../client/src/main/scala/OT.scala}
\\\\

\textbf{Tests}\\
\LARGE TODO
\end{document}

